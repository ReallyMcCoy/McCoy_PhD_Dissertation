\chapter{Introduction}
\label{introduction}

Signals abound in nature. From dense forest to deep ocean, ecosystems buzz with chemicals, calls, and colors in a cacophony of communication. In this dissertation, I focus on signals in two domains: mate choice and pregnancy. My work on the physical makeup and evolutionary purpose of signals has helped unravel secrets about earth’s biodiversity. Beyond basic science, my research touches three other areas: colorful animals inspire new solar power technologies, hormone signals help explain medical complications of human pregnancy, and animal behavior reveals what is going on inside non-human minds. Below, I will give a brief background to the themes I cover in my dissertation. Taken together, the seven chapters illustrate the unofficial name of David Haig’s research group: the Fundamental Interconnectedness of All Things.


\large{\newthought{\textcolor{SchoolColor}{1. Colorful Mating Displays}}}

\normalsize Colorful animals, particularly birds, are a model system for evolutionary biology. Two questions stand out. First, why are some animals colorful when it seems safer to be brown? Second, what is the physical, optical basis of fantastic animal colors? In addition to these fundamental questions, by studying how animals manipulate sunlight for their own purposes, we can develop new solar-powered technologies of our own. In Chapters 1-4, I consider colorful signals in elaborate mating displays. I started this work in birds, at Yale, with Rick Prum’s mentorship, and then brought it to my PhD with David Haig. \\

\vspace{5mm}
\noindent{\textcolor{SchoolColor}{1.1 Sexual Selection}}

For many years, biologists have tried to explain why some animals are brilliantly colorful (often only in one sex). There are three major theories. First, the choosing sex may have aesthetic preferences that could be arbitrary\cite{Prum2012}, shaped by Fisherian runaway selection \cite{Fisher1999}, or influenced by selection on another domain such as foraging (sensory bias) \cite{Dawkins1996}. Second, bright colors could help organisms identify mates of the same species rather than fruitlessly mating with the wrong species \cite{Hill2015}. Third, and most commonly discussed, “honest signaling theory” posits that color is an honest signal of quality, either as a costly signal (e.g., due to parasite load \cite{Folstad1992} or general handicap \cite{Zahavi1975}) or as an index of health\cite{Cantarero2019, Cantarero2017, Hill2019, Simons2012, Weaver2018}.  All of these sexual selective pressures likely operate in concert with the demands of natural selection \cite{Endler1984, McQueen2019}.

Counter to honest signaling theory, evolutionary game theory predicts that males have an incentive to appear better than they are \cite{Dawkins1991, Dawkins1978, Fisher1999, Hill1994, Krebs1984}. Carotenoid pigments—the red, orange, and yellow objects of my research in Chapter 4—are a textbook example of honest signaling \cite{Weaver2018} because they must be eaten rather than synthesized and have some links to metabolic or immunological activity \cite{Cantarero2017, Koch2018, Simons2012, Weaver2017, Weaver2018}. In Chapter 4, I study carotenoid-colored tanager birds (\emph{Ramphocelus} sp.) and show that mate choice is subject to subtle deceptions (consistent with the predictions of evolutionary game theory rather than honest signaling theory). 

\vspace{5mm}
\noindent{\textcolor{SchoolColor}{1.2 Physics of Color}}

Color in nature arises from either pigments or structures. Pigments make color by absorbing certain wavelengths of light but reflecting others. In birds, the most common pigments are melanins (blacks, browns, and reddish-browns) and carotenoids (reds, oranges, and yellows), but some birds also have porphyrins (red, brown, green, and pink) and psittacofulvins (yellow, green, and red) \cite{Cuthill2017, Hill2006, Hill2006a}. Structures make color by physically interfering with light. Usually, researchers study nanostructures \cite{Prum2006}, features that range in size from about 10nm to 700nm. Nanostructures cause iridescent colors and usually cause blue colors in nature. In birds, lattice-like arrangements of melanosomes make feathers iridescent \cite{Doucet2006, Stavenga2018}, like a peacock’s tail, and spongy keratin matrices underpin blue color in many birds \cite{Dufresne2009}.

While pigments and nanostructures are the most commonly-studied drivers of color, there is a third physical mechanism—the focus of my research in Chapters 1-4—that is comparatively understudied: microstructures between 1 µm and 1 mm in size. A handful of studies show that microstructures are important in plant and animal color, from emerald-colored cuckoos  \cite{Harvey2013} to glossy cassowaries\cite{Eliason2020}; from saturated flower petals \cite{Gorton1996} to velvet-black snakes \cite{Spinner2013}. In Chapters 1-3, I show that microstructures play a major role in producing super black color in birds and spiders (reflecting less than 1\% of light). Further, these antireflective signals tell us about the receiver’s sensory experience. I propose that super black color creates an optical illusion fundamental to many animals with color vision, from birds to spiders. 

\vspace{5mm}
\noindent{\textcolor{SchoolColor}{1.3 Bio-inspired Technology}}

By studying the physical cause of unusual coloration in animals, scientists can inspire new technologies for optical equipment, thermal technology, textile design, and solar panels\cite{Dou2020}. For example, butterflies of the genus \emph{Pieris} helped improve solar panel output by more than 40\%. \emph{Pieris} butterflies have ultra-white, lightweight wings \cite{Stavenga2004}; before taking flight, they bask with angled wings to focus light onto their body and warm their flight muscles \cite{Kingsolver1985}. In an ingenious experiment, researchers mounted actual \emph{Pieris} butterfly wings onto solar panels—and found that the  lightweight butterfly wings concentrated light and increased power output by 42.3\%  \cite{Shanks2015}, a gigantic improvement over heavy, expensive metal concentrators \cite{Barber2008, Tang2011}. Likewise, our work on super black spiders in Chapter 2 led my colleague Nikolaj Mandsberg and his team to fabricate spider-inspired arrays for solar panels (with promising early results). 

We face twin perils on the road to sustainability: the catastrophic loss of biodiversity and the looming threat of climate change. Every time a species is lost, we lose something ineffable and indescribably important. We lose critical ecosystem services. But we also lose the tangible, direct value of bioinspiration for sustainable engineering. We should learn all we can about how wild animals harness solar power, because our future—just like theirs—might depend on it.

\large{\newthought{\textcolor{SchoolColor}{2. Pregnancy in Humans and Other Mammals}}}

\normalsize 
Pregnancy is the only time of human life where two (or more) distinct genetic individuals — with their own evolutionary agendas— cohabitate the same body. While many features of pregnancy are joint adaptations, benefiting both mother and embryo, some adaptations of the embryo during pregnancy favor its own development even at a health risk to the mother \cite{Haig1993}, akin to resource wars between humans and our microbiota \cite{Wasielewski2016}. Mothers face a fundamental evolutionary tradeoff: they may either invest more in one offspring or save resources to invest in other offspring. A given embryo values its own health over that of its siblings. This “altercation of generations” \cite{Haig1996} can harm maternal and/or fetal health.

\vspace{5mm}
\noindent{\textcolor{SchoolColor}{2.1 Embryo Selection}}

Most mammalian mothers invest substantial resources in their offspring before and after birth. Therefore, it is worth their while to assess embryo quality and terminate sub-par embryos early \cite{Buchholz1922, Haig1990}.  In all of human life history the highest mortality occurs in the first month after conception \cite{Roberts1975, Wilcox1988}, and embryos must pass signaling “checkpoints” \cite{Ewington2019} in order to be  carried to term. Embryos audition for the role of a lifetime \cite{Forsdyke2019} and mothers are unforgiving judges.  One way that maternal bodies select healthy embryos is by assessing hormonal output. Selection favors mothers who gestate high-quality offspring, but selection favors embryos who can amplify their apparent quality \cite{Haig2019}. The placenta-maternal interface is “a battleground that has shaped remarkable rates of evolutionary change” \cite{Roberts1996}. Beyond placental mammals, many marsupials \cite{Nelson2003} and plants \cite{Lamont1988, Vaughton1993} audition embryos and neonates.

In Chapter 5, I show that hormone signals during pregnancy can be thought of as a coevolutionary arms race where placentas produce higher and higher amounts of hormones and maternal bodies raise their threshold for response \cite{Haig1996}. Placentas show “bewildering” diversity in structure and physiology across mammals \cite{Roberts1996}, akin to the diversity of birds-of-paradise.

\vspace{5mm}
\noindent{\textcolor{SchoolColor}{2.2 Sibling Competition}}

Siblings compete intensely in many taxa \cite{Hudson2007, Mock1997, Roulin2012, Royle1999}; e.g., siblings fight over space in the uterus \cite{Chapman2013}, access to nipples \cite{Golla1999}, and alloparental care \cite{Neuenschwander2003}. Pronghorn fetuses (\emph{Antilocapra americana}) implant in groups of 5-8 in each of two uterine horns in the mother; however, the fetuses grow a “necrotic tip” which impales all fetuses except one per horn \cite{OGara1969}. There is some evidence that competition is stiffer between siblings of the same sex; in spotted hyenas (\emph{Crocuta crocuta}) same-sex litters have higher rates of aggression than do mixed-sex litters during infancy \cite{Golla1999} and same-sex human siblings have lower reproductive success \cite{Gibson2011, Ji2013, Nitsch2012} (but see \cite{Butikofer2019}). These sibling conflicts can have major impacts on health and reproductive success. In Chapter 6, I find evidence that having siblings, particularly same-sex littermates, harms lifespan and reproduction in a peculiar bunch of monkeys (the Callitrichidae).

\vspace{5mm}
\noindent{\textcolor{SchoolColor}{1.3 Pathologies of Pregnancy}}

It is a seeming paradox that pregnancy-- required for the furtherance of human life-- so often risks the life of mother and/or offspring. In humans, healthy pregnancies cause significant physical stress; pathological pregnancies can lead to severe bleeding, diabetes, high blood pressure, and death. In 2017, the World Health Organization reported that the lifetime risk of maternal death worldwide was 1 in 190. Traditionally, it is argued that pregnancy complications are the unintended evolutionary by-products of having large heads (we are smart) but narrow hips (we are bipedal). However, leading causes of pregnancy-related maternal deaths – such as hemorrhage and high blood pressure \cite{Say2014} -- cannot be explained by head and hip size. Instead, the most common complications of pregnancy result from parent-offspring conflict over resources: hemorrhage, high blood pressure, and gestational diabetes are overexpressions of embryonic adaptations to promote more nutrient-rich blood flow to the placenta \cite{Haig1993}. In Chapters 5-6, I examine health complications of pregnancy in mammals. The lens of conflict explains why the best treatment for some life-threatening maternal problems, such as preeclampsia, is to deliver the baby. 

\large{\newthought{\textcolor{SchoolColor}{3. Animal Minds}}}

\normalsize 
In the preceding two sections of this introduction, I have described how we can study signals to learn something about evolution. How hormones and receptors evolve during pregnancy tells us about parent-offspring conflict; the optical basis of signals in birds and spiders tells us about honesty and deception in mate choice. But these studies only make inferences about animal minds. What are the females actually feeling when they see a brilliantly-colored male framed by super black? How do genomic conflicts (say, between paternally-imprinted and maternally-imprinted genes) translate into subjective experience of a divided self \cite{Haig2020}? I hope to tackle these specific questions in my future work, but a related question about the subjective experience of New Caledonian crows (\emph{Corvus moneduloides}) forms the final chapter of my dissertation. I first learned about the rich inner lives of animals thanks to two mentors: Laurie Santos at Yale and a parrot named Penny who taught me how to speak. After several summers of fieldwork on the unforgettable island of New Caledonia, I can contribute some experimental data about the inner lives of animals.


\vspace{5mm}
\noindent{\textcolor{SchoolColor}{3.1 Testing Animal Mood}}

Animals cannot tell us how they are feeling. However, cognitive scientists have developed a clever tool to peek into the subjective experience of animals based on the “glass half-full” paradigm. Happy people see a half-full glass where sadder people see the same glass as half-empty. Our mood—our subjective feelings—influence our interpretation of ambiguous stimuli (the glass). Likewise, we know from substantial past experimental work that animals in a better mood are more optimistic about ambiguous stimuli\cite{Bateson2011, Burman2008, Douglas2012, Eysenck1991, Harding2004, Mendl2009, Mendl2010a, Mendl2010, Paul2005}(see Chapter 7 for experimental details). Previously, most studies of animal mood have tested how circumstances, or manipulation, influence animals. For example, starlings \cite{Bateson2007, Matheson2008} and pigs \cite{Douglas2012} are in a better mood when housed with enrichment. Calves separated from their mothers \cite{Daros2014}, socially isolated chicks \cite{Salmeto2011}, and shaken-up flies \cite{Deakin2018} are pessimistic. More happily, you can put rats in a good mood by tickling them \cite{Rygula2012}. 

Circumstances make animals happy or sad, but a rich life is defined not only by circumstantial comfort. Human welfare depends on a sense of purpose and meaningful intellectual stimulation. In my final chapter, I show that wild New Caledonian Crows, like humans, enjoy tool use for the sake of tool use—akin to humans who enjoy playing bridge or solving crosswords. The crows are happier after they have used a tool compared to control conditions. 

At the end of the experimental study, we released the birds back into the wild. I like to imagine that they tell their friends about their strange winter—the winter when they ate steak three times a day, stole bottle caps from scientists, and taught some humans how to use tools. 

And I hope, and think, that the crows lived \textcolor{SchoolColor}{Happily Ever After}.

 