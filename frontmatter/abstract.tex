% the abstract

Animals spend their lives sending and receiving signals. Before birth, embryos converse with mothers through hormones and receptors; in adulthood, many animals attract and assess partners through colorful mating dances. In this dissertation, I used experimental optics and evolutionary theory to analyze the physical makeup and evolutionary purposes of animal signals—from placentas to birds-of-paradise. \textcolor{SchoolColor}{\textbf{In Chapter 1}}, I show that ornate male birds-of-paradise evolved “super black” feathers with microstructures that absorb up to 99.95\% of light. Super black is an evolved optical illusion which makes males’ bright colors appear brighter, even glowing, to observing females. \textcolor{SchoolColor}{\textbf{In Chapter 2}}, I find that colorful peacock spiders—the alluring arachnid analogue of birds-of-paradise—have super black microlens arrays, bumps that are optimally sized and shaped to absorb more, and reflect less, light. \textcolor{SchoolColor}{\textbf{In Chapter 3}}, I demonstrate that super black evolved convergently in 15 families of sexually-selected, brightly-colored birds. Super black is not the only adaptation to harness physical rules of light for evolutionary purposes. \textcolor{SchoolColor}{\textbf{In Chapter 4}}, I show that some carotenoid-colored male birds (an archetypal example of honest signaling) fly under false colors: male tanagers use microstructures to deceptively amplify their appearance. These lessons about deception in mate choice also apply to mammalian pregnancy, where selective mothers evaluate embryos and automatically terminate low-quality embryos (akin to selective female birds choosing a mate). \textcolor{SchoolColor}{\textbf{In Chapter 5}}, I describe evidence that human embryos exaggerate their own quality to satisfy choosy mothers: in primates and convergently in horses, embryos produce signalling hormones (i) in massive quantities and (ii) with biochemical changes to extend half-life (two evolutionary signatures of deception and escalation). Beyond maternal-embryo conflict, siblings can also clash during pregnancy. \textcolor{SchoolColor}{\textbf{In Chapter 6}}, I  analyze a large dataset (n = 27,080) of marmoset monkey births to show that conflicts of interest between same-sex co-gestating siblings harm health and survivorship. Finally, \textcolor{SchoolColor}{\textbf{in Chapter 7}} I close on an optimistic note by presenting experimental evidence that tool-making New Caledonian crows feel happier after they have used tools, like humans who are intrinsically motivated to play chess or solve crosswords. All of these projects included collaborators to whom I am grateful. 