% the acknowledgments section
My acknowledgements must start with David Haig, my PhD supervisor. After meeting David for the first time, I rushed out of his office full of energy and jotted down the following note: “David Haig: an absolutely awesome guy who studies all sorts of things, puts jokes in his paper, and even writes about stuff like philosophy.” I decided on the spot I would join his group if I could, and, luckily, I did. At our first lunch, David and I sat down and talked for an hour. Many memorable conversations followed, each brimming with tangents, visits to the Oxford English Dictionary, forays into books ancient and new, and lots of laughter. Through his kindness and brilliance, David has fundamentally changed the way I view evolution, life, and mentorship. While I will not be able to pop my head around the corner to say hello anymore, I look forward to many more conversations in our future. Anyone who has not met him: go! Run, at top speed! Have a conversation with David Haig!

I have to convey my immense gratitude to my mentors. First, I am grateful for my brilliant and supportive committee— Catherine Dulac, Scott Edwards, and Naomi Pierce—with whom I have had so many illuminating conversations. Thank you also to George Lauder for his strong support and for our conversations about biophysics. Thank you to Lydia Carmosino for being an organizational genius. My dissertation would not have been possible without Brett Frye, Allison Shultz, and Martina Schiestl. Finally, I am a biologist because Leo Buss showed me the joys of scientific research through the Collections of the Peabody Museum at Yale. Rick Prum and Kristof Zyskowski introduced me to the beauty of birds, and Laurie Santos supported my curiosity about animal minds. And I will not soon forget learning from Chris Norris in the paleontological catacombs.

My labmates, and the members of our weekly FIAT group, greatly enriched my PhD experience both socially and academically. Special thank you to Jenna Kotler, Ava Mainieri, Holly Elmore, Brianna Weir, and Arvid Ågren. In MCZ \#410 the tea kettle (and the conversation) never got cold.

Graduate school would have been a pale imitation of what it was if not for my wonderful cohort! We met during the Snowpocalypse of 2015, got stuck on campus, and then stuck together ever since. Thank you Sam Church, Molly Edwards, Brandon Enalls, Jess Gersony, Ben Goulet-Scott, Eadaoin Harney, Alyssa Hernandez, Nick Herrmann, Milo Johnson, Sang Il Kim, Vanessa Knutson, Phil Lai, Amaneet Lochab, Sofia Prado-Irwin, Jonathan Schmitt, Kristel Schoonderwoerd, Dan Utter, Yang Wang, Brock Wooldridge, James Xue, and Min Ya. 

Friends make life richer. Tom Barron has helped me see more clearly and enjoy the wonders of nature. Sean and Judy Palfrey made Adams House not just a House but a Home. Cellar Thursday was a high point of every week; thank you Liz Asai and the rotating cast that made Thursdays sparkle. Finally, thank to to the friends I've gotten to spend time in Cambridge-- to name just a few, Priyanka deSouza, Ed and Imo Doddridge, Allan Hsiao, Rachel Kolb, Ela Leshem and Doni Bloomfield, Charles Masaki, and Mubeen Shakir. Finally, Voicelab has brought music and joy to my life beyond measure. 

Thank you to my partner, William Burke. You make life so fun and special.

And thank you to my family: Mom (known to the world as Mary L. Marazita), Dad (Richard T. McCoy III), Marty, Tory, Tom, and Sammie. 


%The Museum of Comparative Zoology has been my home. Time travelers: you can find me between 2015 and 2020, through the door between the Emu and the Musk Duck, where the great whale’s tail points. 
